\documentclass{report}
    \usepackage{graphicx} 
    \usepackage{color}

    \definecolor{shubham-blue}{RGB}{66,134,244}
    \definecolor{shubham-red}{RGB}{247,56,56}

    \begin{document}
   
    \begin{titlepage}
        \centering
        \vfill
        \includegraphics[width=10cm,keepaspectratio]{SGSITS-Indore-Logo.png}
        \vfill        
        {\bfseries\Huge
        \textcolor{shubham-blue}{Latex Assignment\\
            Computer Peripheral and Interfaces}
            \vskip2cm
            \textcolor{shubham-red}{Shubham Menkudle}
        }    
        \vfill
    \end{titlepage}

    \part{Assignment 1}
    \chapter{Assignment 1}
    \section{Use Paragraphs, change fontsize and font-family}
    \paragraph{Paragraph 1}
    
    LaTeX, which is pronounced «Lah-tech» or «Lay-tech» (to rhyme with «blech» or «Bertolt Brecht»), is a document preparation system for high-quality typesetting. It is most often used for medium-to-large technical or scientific documents but it can be used for almost any form of publishing.
    \subparagraph{subparagraph 1}
    LaTeX is not a word processor! Instead, LaTeX encourages authors not to worry too much about the appearance of their documents but to concentrate on getting the right content.
    
    \subparagraph{subparagraph 2}
    LaTeX is based on the idea that it is better to leave document design to document designers, and to let authors get on with writing documents.
    \section{Write algorithm for Bubble sort using package algorithm}
    \section{Solve following equation step by step:}
    \section{Multiply two matrices, one 3 by 2 and second 2 by 4. Take matrix value
    yourself}
    \section{Take your own code from an assignment and make it color-coded using
    minted}
    \section{Using chapter, section, subsection, subsubsection etc}
    \section{Write binomial formula for (a + b) 5}
    \section{Write formula for sigma x 2 }
    \section{solve following:}
    \section{Using references for tables, figures, algorithms and citations}
    \section{creating footnotes}
    \section{how to link table of content entries to page numbsers}

    

    \part{Assignment 1 Code}


    \end{document}